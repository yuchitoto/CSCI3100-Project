\begin{document}
\paragraph{Drag and Drop System}~{}
\\
This is the system provides dragging blocks and dropping onto the programming area for user to easily write program.\\
This system contains the following block:
  \begin{itemize}
	\item Variable Block, which is used to store data with a name given by user. The Variable Block contains the following type:
		\begin{itemize}
		\item[$\ast$] Alphabet : only can store one alphabet for each variable.
		\item[$\ast$] String : can store more than one alphabets with declared length of the variable.
		\item[$\ast$] Integer : only can store integer.
		\item[$\ast$] Real Number : can store real number with decimal point.
    \end{itemize}
	\item Assignment Block, which is used to assign the data of one variable to another variable with the same type.
	\item Expression Blocks, which user can type anything in the Blocks.
	\item IF Block, which is a logical block to do logical operations such as AND gate, OR gate and NOT gate. User can also do comparison between two Variable Blocks.
	\item ELSE Block, which is an optional block follow the IF Block and this block will only be run when the logical condition in the IF Block is false.
	\item Loop Block, which provides looping the blocks inside the Loop Block under a logical condition. This block begin with a IF Block and the blocks inside will starting looping until the condition in the IF Block is false.
	\item Input Block, which is used to get data and store to a suitable Variable Block while the program is running.
	\item Output Block, which is used to output text or data during program runtime.
	\item Function Block, which to conclude other Blocks into one big Block. It always appear in pair that one is used to define and other one is used to call itself in the Main Block. It can also define the type of the Function Block like Alphabet, String, Integer and Real Number. User can adapt Variable Block which means when user can call it with passing Variable Block into the Block.
	\item Main Block, which is used to run the Blocks inside in the program. It will appear in the programming area automatically. User can also delete it.
	\end{itemize}




\end{document}
